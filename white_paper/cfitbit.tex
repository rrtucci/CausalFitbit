

\documentclass[12pt]{article}
\usepackage{graphicx}     %standard package. Note graphics<graphicx<epsfig
\usepackage{color}     %standard package
\usepackage{hyperref} %standard package written by Sebastian Rahtz
%\usepackage{subeqn}    %allows equations 1a, 1b
%\usepackage{subfig}    %allows figures 1a, 1b
\usepackage{amsmath}
\usepackage{amssymb}
\usepackage{xurl}
\usepackage{bbm}%for indicator function
%\bibliographystyle{alpha}
\usepackage{floatrow}


\usepackage[toc,page]{appendix}
\usepackage[nottoc]{tocbibind}

\usepackage[color,matrix,frame,arrow,curve]{xy}
\usepackage[table,xcdraw]{xcolor}
\usepackage[shortlabels]{enumitem}
\usepackage{array}
\usepackage{pdflscape}
\usepackage{multicol, multirow}
\usepackage{ctable}
\usepackage{dirtree}
\usepackage{subfig}
\usepackage{longtable}

%\usepackage{fdsymbol}
%\usepackage{animate}% play animation

\usepackage[vlined,ruled]{algorithm2e}
\newcommand{\mycommfont}[1]{\small\ttfamily\textcolor{blue}{#1}}
\SetCommentSty{mycommfont}

%highlighting
\usepackage{mdframed}

\usepackage[makeroom]{cancel}

\usepackage{pdfpages}
\usepackage{hyperxmp}

\paperheight=11in
\topmargin=0in
\headheight=0in
\headsep=0in
\topskip=0in
\textheight=8.5in
\footskip=.5in


\paperwidth=8.5in
\oddsidemargin=.25in
\evensidemargin=.25in
\textwidth=6.0in
\parindent=.5in

\newcommand{\chapquote}[3]{\begin{quotation} \textit{#1} \end{quotation} \begin{flushright} - #2, \textit{#3}\end{flushright} }

\newtheorem{claim}{Claim}
\newcommand{\proof}[0]{{\bf proof:} }
\newcommand{\qed}[0]{\quad\newline \noindent{\bf QED }}


\newcommand{\bra}[1]{\langle#1|}
\newcommand{\ket}[1]{|#1\rangle}
\newcommand{\av}[1]{\left\langle#1\right\rangle}
\newcommand{\pder}[2]{\frac{\partial#1}{\partial#2}}
\newcommand{\der}[2]{\frac{d#1}{d#2}}
\newcommand{\tr}[0]{{\rm tr }}
\newcommand{\beq}{\begin{equation}}
\newcommand{\eeq}{\end{equation}}
\newcommand{\bsub}{\begin{subequations}}
	\newcommand{\esub}{\end{subequations}}
\newcommand{\beqa}{\begin{eqnarray}}
\newcommand{\eeqa}{\end{eqnarray}}
\newcommand{\rarrow}[0]{\rightarrow}
\newcommand{\larrow}[0]{\leftarrow}
\newcommand{\uarrow}[0]{\uparrow}
\newcommand{\darrow}[0]{\downarrow}
\newcommand{\Rarrow}[0]{\Rightarrow}
\newcommand{\nRarrow}[0]{\nRightarrow}
\newcommand{\Larrow}[0]{\Leftarrow}
\newcommand{\nLarrow}[0]{\nLeftarrow}
\newcommand{\ul}[1]{\underline{#1}}
\newcommand{\ol}[1]{\overline{#1}}
\newcommand{\ZZ}[0]{ {\mathbb{Z}}}
\newcommand{\RR}[0]{{ \mathbb{R}} }
\newcommand{\CC}[0]{{ \mathbb{C}} }
\newcommand{\XX}[0]{{ \mathbb{X}} }
\newcommand{\ground}{{}_{\stackrel{\stackrel{\displaystyle{\bot}}{-}}{.}}}
\newcommand{\norm}[1]{\parallel#1\parallel}
\newcommand{\eqdef}[0]{\;\;\stackrel{\text{def}}{=}\;\;}

\newcommand{\heavy}[0]{\mathbf{H}}

\newcommand{\rva}[0]{{\ul{a}}}
\newcommand{\rvb}[0]{{\ul{b}}}
\newcommand{\rvc}[0]{{\ul{c}}}
\newcommand{\rvd}[0]{{\ul{d}}}
\newcommand{\rve}[0]{{\ul{e}}}
\newcommand{\rvf}[0]{{\ul{f}}}
\newcommand{\rvg}[0]{{\ul{g}}}
\newcommand{\rvh}[0]{{\ul{h}}}
\newcommand{\rvi}[0]{{\ul{i}}}
\newcommand{\rvj}[0]{{\ul{j}}}
\newcommand{\rvk}[0]{{\ul{k}}}
\newcommand{\rvl}[0]{{\ul{l}}}
\newcommand{\rvll}[0]{{\ul{\ell}}}
\newcommand{\rvm}[0]{{\ul{m}}}
\newcommand{\rvn}[0]{{\ul{n}}}
\newcommand{\rvo}[0]{{\ul{o}}}
\newcommand{\rvp}[0]{{\ul{p}}}
\newcommand{\rvq}[0]{{\ul{q}}}
\newcommand{\rvr}[0]{{\ul{r}}}
\newcommand{\rvs}[0]{{\ul{s}}}
\newcommand{\rvt}[0]{{\ul{t}}}
\newcommand{\rvu}[0]{{\ul{u}}}
\newcommand{\rvv}[0]{{\ul{v}}}
\newcommand{\rvw}[0]{{\ul{w}}}
\newcommand{\rvx}[0]{{\ul{x}}}
\newcommand{\rvy}[0]{{\ul{y}}}
\newcommand{\rvz}[0]{{\ul{z}}}


\newcommand{\rvA}[0]{{\ul{A}}}
\newcommand{\rvB}[0]{{\ul{B}}}
\newcommand{\rvC}[0]{{\ul{C}}}
\newcommand{\rvD}[0]{{\ul{D}}}
\newcommand{\rvE}[0]{{\ul{E}}}
\newcommand{\rvF}[0]{{\ul{F}}}
\newcommand{\rvH}[0]{{\ul{H}}}
\newcommand{\rvK}[0]{{\ul{K}}}
\newcommand{\rvL}[0]{{\ul{L}}}
\newcommand{\rvN}[0]{{\ul{N}}}
\newcommand{\rvQ}[0]{{\ul{Q}}}
\newcommand{\rvR}[0]{{\ul{R}}}
\newcommand{\rvT}[0]{{\ul{T}}}
\newcommand{\rvU}[0]{{\ul{U}}}
\newcommand{\rvV}[0]{{\ul{V}}}
\newcommand{\rvW}[0]{{\ul{W}}}
\newcommand{\rvX}[0]{{\ul{X}}}
\newcommand{\rvY}[0]{{\ul{Y}}}
\newcommand{\rvZ}[0]{{\ul{Z}}}

\newcommand{\rvxi}[0]{{\ul{\xi}}}
\newcommand{\rvzeta}[0]{{\ul{\zeta}}}
\newcommand{\rvtheta}[0]{{\ul{\theta}}}
\newcommand{\rvmu}[0]{{\ul{\mu}}}
\newcommand{\rvsig}[0]{{\ul{\sigma}}}
\newcommand{\rvDel}[0]{{\ul{\Delta}}}
\newcommand{\rvtau}[0]{{\ul{\tau}}}

\newcommand{\cala}[0]{{\cal A}}
\newcommand{\calb}[0]{{\cal B}}
\newcommand{\calc}[0]{{\cal C}}
\newcommand{\cald}[0]{{\cal D}}
\newcommand{\cale}[0]{{\cal E}}
\newcommand{\calf}[0]{{\cal F}}
\newcommand{\calg}[0]{{\cal G}}
\newcommand{\calh}[0]{{\cal H}}
\newcommand{\cali}[0]{{\cal I}}
\newcommand{\calk}[0]{{\cal K}}
\newcommand{\call}[0]{{\cal L}}
\newcommand{\calm}[0]{{\cal M}}
\newcommand{\caln}[0]{{\cal N}}
\newcommand{\calo}[0]{{\cal O}}
\newcommand{\calp}[0]{{\cal P}}
\newcommand{\calq}[0]{{\cal Q}}
\newcommand{\calr}[0]{{\cal R}}
\newcommand{\cals}[0]{{\cal S}}
\newcommand{\calt}[0]{{\cal T}}
\newcommand{\calu}[0]{{\cal U}}
\newcommand{\calv}[0]{{\cal V}}
\newcommand{\calw}[0]{{\cal W}}
\newcommand{\calx}[0]{{\cal X}}
\newcommand{\caly}[0]{{\cal Y}}
\newcommand{\calz}[0]{{\cal Z}}

\newcommand{\Pmat}[4]{\calp\left[
\begin{array}{cc}#1&#2\\#3&#4
\end{array}\right]}

%\newcommand{\PN}[0]{PN^{0,0}_{1,1}}
%\newcommand{\PS}[0]{PS^{1,1}_{0,0}}
\newcommand{\PN}[0]{PN}
\newcommand{\PS}[0]{PS}

\newcommand{\lam}[0]{\lambda}
\newcommand{\Lam}[0]{\Lambda}
\newcommand{\alp}[0]{\alpha}
\newcommand{\eps}[0]{\epsilon}
\newcommand{\s}[0]{\sigma}
\newcommand{\su}[0]{{\Sigma}}

\newcommand{\pp}[0]{\mathbb{P}}
\newcommand{\dbm}[0]{{
		[1,\partial_{b},\partial_{m}]
}}

\newcommand{\dg}[0]{{[1, \partial_{\theta_G}]}}
\newcommand{\dd}[0]{{[1, \partial_{\theta_D}]}}
\newcommand{\dgd}[0]{{[1, \partial_{\theta_G}, \partial_{\theta_D}]}}

\newcommand{\veca}[0]{{\vec{a}}}
\newcommand{\vecb}[0]{{\vec{b}}}
\newcommand{\vecc}[0]{{\vec{c}}}
\newcommand{\vecd}[0]{{\vec{d}}}
\newcommand{\vecf}[0]{{\vec{f}}}
\newcommand{\vech}[0]{{\vec{h}}}
\newcommand{\vecr}[0]{{\vec{r}}}
\newcommand{\vecs}[0]{{\vec{s}}}
\newcommand{\vecu}[0]{{\vec{u}}}
\newcommand{\vecx}[0]{{\vec{x}}}
\newcommand{\vecy}[0]{{\vec{y}}}
\newcommand{\vechy}[0]{{\vec{\haty}}}
\newcommand{\vtheta}[0]{{\vec{\theta}}}

\newcommand{\haty}[0]{{\widehat{y}}}
\newcommand{\hatx}[0]{{\widehat{x}}}
\newcommand{\hata}[0]{{\widehat{a}}}
\newcommand{\hatr}[0]{{\widehat{r}}}


\newcommand{\cond}[0]{{\:\mathbf{|}\:}}
\newcommand{\mymathbf}[1]{#1}

\newcommand{\ranvec}[1]{\ul{\vec{#1}}}
\newcommand{\indi}[0]{\mathbbm{1}}
\newcommand{\smoid}[0]{{\rm smoid}}
\newcommand{\lodds}[0]{{\rm lodds}}
\newcommand{\expit}[0]{{\rm expit}}
\newcommand{\logit}[0]{{\rm logit}}
\newcommand{\sign}[0]{{\rm sign}}

\renewcommand{\labelitemii}{$\bullet$}

\newcommand{\HAT}[1]{{\widehat{#1}}}
\newcommand{\TIL}[1]{{\widetilde{#1}}}
\newcommand{\tild}[0]{{\TIL{d}}}
\newcommand{\tile}[0]{{\TIL{e}}}
\newcommand{\tilg}[0]{{\TIL{g}}}
\newcommand{\tilu}[0]{{\TIL{u}}}
\newcommand{\tilx}[0]{{\TIL{x}}}
\newcommand{\tilP}[0]{{\TIL{P}}}
\newcommand{\tilPT}[0]{{\TIL{P}_\theta}}

\newcommand{\maparrow}[1]
{\xymatrix{\ar[r]_{#1}&}}


\newcommand{\ucalm}[0]{\ul{\calm}}

\newcommand{\bool}[0]{\{0,1\}}

\newcommand{\argmin}{\mathop{\mathrm{argmin}}\limits}
\newcommand{\argmax}{\mathop{\mathrm{argmax}}\limits}

\newcommand{\softmax}[0]{{\rm softmax}}

\newcommand{\A}[0]{\wedge}
\newcommand{\V}[0]{\vee}
\newcommand{\xor}{\oplus}
\newcommand{\bigA}[0]{\bigwedge}
\newcommand{\bigV}[0]{\bigvee}
\newcommand{\bigxor}{\bigoplus}


\newcommand{\rdart}[0]{\Rightarrow}
\newcommand{\ldart}[0]{\Leftarrow}
\newcommand{\rveps}[0]{\ul{\eps}}

\newcommand{\hatvar}[0]{\widehat{\sigma^2}}
\newcommand{\ptp}[0]{{(t)}}

\newcommand{\tseries}[1]{{\{#1\}_{\forall t}}}

\newcommand{\xbeta}[0]{X_\s^T\beta}
\newcommand{\xtau}[0]{X_\s^T\tau}

%arguments phi1, phi2, phi3, e
\newcommand{\rbd}[4]{
\xymatrix@-1.3pc{
&#1\ar[d]&&#2\ar[d]
\\
&\rvx_1\ar[rr]
&&
\rvx_2
\ar `r[rd][rd]
\\
0\ar`u[u][ru]
\ar`d[dr][rrd]
&&&&\rvA\ar[r]&#4
\\
&&
\rvx_3
\ar `r[rru][rru]
\\
&&#3\ar[u]
}
}

\newcommand{\rulezeroif}[0]{
If $(\rvb. \perp \rva.
|\rvr., \rvs.)$
in $G$, then}

\newcommand{\rulezerothen}[0]{
$\rva.=a. \leftrightarrow 1$}

\newcommand{\rulezeroP}[0]{
P(b.|a.,r.,s.)=P(b.|r., s.)}

\newcommand{\rulezeroH}[0]{
H(\rvb.:\rva.|\rvr.,\rvs.)=0}

\newcommand{\rulezeropic}[0]{
\xymatrix@C=1pc@R=1pc{
&r.\ar[dr]
&s.\ar[d]
\\
a.\ar[rr]
&&b.
&=
}
\xymatrix@C=1pc@R=1pc{
&r.\ar[dr]&s.\ar[d]
\\
a.\ar[rr]|0
&&b.
}}

\newcommand{\ruleoneif}[0]{
If $(\rvb. \perp \rva.
|\rvr., \rvs.)$
in $\cald_{\rvr.} G$, then}

\newcommand{\ruleonethen}[0]{
$\rva.=a. \leftrightarrow 1$}

\newcommand{\ruleoneP}[0]{
P(b.|a., \cald\rvr.=r.,s.)=
P(b.|\cald\rvr.=r., s.)}

\newcommand{\ruleoneH}[0]{
H(\rvb.:\rva.|\cald\rvr.,\rvs.)=0}

\newcommand{\ruleonepic}[0]{
\xymatrix@C=1pc@R=1pc{
&\cald\rvr.=r.\ar[dr]
&s.\ar[d]
\\
a.\ar[rr]
&&b.
&=
}
\xymatrix@C=1pc@R=1pc{
&\cald\rvr.=r.\ar[dr]
&s.\ar[d]
\\
a.\ar[rr]|0
&&b.
}}

\newcommand{\ruletwoif}[0]{
If $(\rvb.\perp \rva. |
 \rvr., \rvs.)$
in $\call_{\rva.}\cald_{\rvr.} G$,
 then}

\newcommand{\ruletwothen}[0]{
$\cald \rva.=a. \leftrightarrow \rva.=a.$}

\newcommand{\ruletwoP}[0]{
P(b.|\cald\rva.=a., \cald\rvr.=r., s.)=
P(b.|a., \cald\rvr.=r.,  s.)}

\newcommand{\ruletwoH}[0]{
H(\rvb.:\cald\rva.|\cald\rvr.,  \rvs.)
=
H(\rvb.:\rva.|\cald\rvr.,  \rvs.)}

\newcommand{\ruletwopicA}[0]{
\xymatrix@C=1pc@R=1pc{
\;\ar[d]_0
&\cald\rvr.=r.\ar[dr]
&s.\ar[d]
\\
\cald\rva.=a.\ar[rr]\ar[d]
&&b.
&=\quad\quad
\\
&
}
\xymatrix@C=1pc@R=1pc{
\;\ar[d]_0
&&\cald\rvr.=r.\ar[dr]
&s.\ar[d]
\\
\cald\rva.=a.\ar[d]
&a.\ar[rr]
&&b.
\\
&
}}
\newcommand{\ruletwopicB}[0]{
\xymatrix@C=1pc@R=1pc{
\;\ar[d]
&\cald\rvr.=r.\ar[dr]
&s.\ar[d]
\\
a.\ar[rr]\ar[d]
&&b.
&=\quad\quad
\\
&
}
\xymatrix@C=1pc@R=1pc{
\;\ar[d]
&&\cald\rvr.=r.\ar[dr]
&s.\ar[d]
\\
a.\ar[d]
&\cald\rva.=a.\ar[rr]
&&b.
\\
&
}}

\newcommand{\rulethreeif}[0]{
If $
(\rvb. \perp \rva.
| \rvr., \rvs.)$
in $\cald_{\rva.-an(\rvs.)}
\cald_{\rvr.}G$,
 then}

\newcommand{\rulethreethen}[0]{
$\cald \rva.=a. \leftrightarrow 1$}

\newcommand{\rulethreeP}[0]{
P(b.|\cald\rva.=a.,\cald\rvr.=r.,  s.)=
P(b.|\cald\rvr.=r., s.)}

\newcommand{\rulethreeH}[0]{
H(\rvb.:\cald\rva.|\cald\rvr., \rvs.)=0}

\newcommand{\rulethreepic}[0]{
\xymatrix@C=1pc@R=1pc{
&
&\cald\rvr.=r.\ar[dr]
&s.\ar[d]
\\
&\cald\rva.=a.\ar[rr]
&&b.
&=
}
\xymatrix@C=1pc@R=1pc{
&
&\cald\rvr.=r.\ar[dr]
&s.\ar[d]
\\
&\cald\rva.=a.\ar[rr]|0
&&b.
}}

\newcommand{\bdoordef}[0]{
Suppose that we have access to data
that allows us to
estimate a probability
distribution
 $P(x., y., z.)$.
Hence, the variables
$\rvx., \rvy., \rvz.$ are
ALL observed (i.e, not hidden).
Then we say that the
backdoor $\rvz.$
satisfies the
{\bf backdoor adjustment criterion}
relative to $(\rvx., \rvy.)$
if
\begin{enumerate}
\item
All backdoor paths from $\rvx.$
to $\rvy.$
 are blocked by  conditioning on
 $\rvz.$.
\item
$\rvz. \cap de(\rvx.)=\emptyset$.
\end{enumerate}
}

\newcommand{\bdoorclaim}[0]{
If $\rvz.$ satisfies the
backdoor criterion relative to
 $(\rvx., \rvy.)$, then

\beqa
P(y. | \cald \rvx.=x.)&=&
\sum_{z.} P(y.|x., z.)P(z.)
\\
&=&
\begin{array}{l}
\\
\\
\end{array}
\xymatrix{
\sum z.\ar[dr]
\\
x.\ar[r]
&y.
}
\eeqa
where $\sum z.$ means node
$\rvz.$ is summed over.
}

\newcommand{\fdoordef}[0]{
Suppose that we have access to data
that allows us to
estimate a probability
distribution
 $P(x., m., y.)$.
Hence, the variables
$\rvx., \rvm., \rvy.$ are
ALL observed (i.e, not hidden).
Then we say that
the frontdoor $\rvm.$
satisfies the
{\bf frontdoor adjustment criterion}
relative to $(\rvx., \rvy.)$
if
\begin{enumerate}
\item
All directed paths from
$\rvx.$ to $\rvy.$ are intercepted by
(i.e., have a node in) $\rvm.$.
\item
All backdoor paths from $\rvx.$ to
$\rvm.$ are blocked.
\item
All backdoor paths from
on $\rvm.$ to $\rvy.$
are blocked by conditioning
on  $\rvx.$.
\end{enumerate}
}

\newcommand{\fdoorclaim}[0]{
If $\rvm.$ satisfies the
frontdoor criterion
relative to $(\rvx., \rvy.)$,
and $P(x.,m.)>0$, then

\beqa
P(y. | \cald \rvx.=x.)&=&\sum_{m.}
\underbrace{\left[\sum_{x'.}
P(y.|x'., m.)P(x'.)\right]}_
{P(y.|\cald \rvm.=m.)}
\underbrace{P(m.|x.)}_
{P(m.|\cald \rvx.=x.)}
\\
&=&
\xymatrix{
&\sum x'.\ar[dr]
\\
x.´\ar[r]
&\sum m.\ar[r]&y.
}
\eeqa
where $\sum x'.$ and
$\sum m.$
means nodes
$\rvx'.$ and $\rvm.$
are summed over.
}


%Symmetry
\newcommand{\symrule}[0]{
$\rva\perp_P\rvb\implies \rvb\perp_P\rva$}

\newcommand{\symruleH}[0]{
$H(\rva:\rvb)=0\implies H(\rvb:\rva)=0$}

%Decomposition
\newcommand{\decrule}[0]{
$\rva\perp_P\rvb, \rvc\implies
\rva\perp_P\rvb \text{ and } \rva\perp_P\rvc$}

\newcommand{\decruleH}[0]{
$H(\rva:\rvb, \rvc)=0\implies
H(\rva:\rvb)=0 \text{ and } H(\rva:\rvc)=0$}

%Weak Union
\newcommand{\wearule}[0]{
$\rva\perp_P \rvb, \rvc \implies
\rva\perp_P\rvb|\rvc\text{ and }\rva\perp_P\rvc|\rvb$}

\newcommand{\wearuleH}[0]{
$H(\rva:\rvb, \rvc)=0 \implies
H(\rva:\rvb|\rvc)=0\text{ and }H(\rva:\rvc|\rvb)=0$}

%Contraction
\newcommand{\conrule}[0]{
$\rva\perp_P\rvb|\rvc\text{ and }\rva\perp_P \rvc
\implies \rva\perp_P \rvb, \rvc$}

\newcommand{\conruleH}[0]{
$H(\rva:\rvb|\rvc)=0\text{ and }H(\rva:\rvc)=0
\implies H(\rva:\rvb, \rvc)=0$}

%Intersection
\newcommand{\intrule}[0]{
$\rva\perp_P\rvb|\rvc, \rvd\text{ and }
\rva\perp_P \rvd|\rvc, \rvb\implies
\rva\perp_P \rvb,\rvd|\rvc$}

\newcommand{\intruleH}[0]{
$H(\rva:\rvb|\rvc, \rvd)=0\text{ and }
H(\rva:\rvd|\rvc, \rvb)=0\implies
H(\rva:\rvb,\rvd|\rvc)=0$}

\newcommand{\dotbarmu}[0]{{\cdot|\mu}}
\newcommand{\dotmu}[0]{{\cdot, \mu}}
\newcommand{\kbarmu}[0]{{k|\mu}}
\newcommand{\kmu}[0]{{k,\mu}}
\newcommand{\plusbarmu}[0]{{+|\mu}}
\newcommand{\plusmu}[0]{{+,\mu}}

\newcommand{\bnlearn}[0]{{\tt bnlearn\;}}

\newcommand{\sqsig}[0]{{[\sigma]}}

\newcommand{\misscellone}[0]{
\begin{array}{c}
\frac{1}{nsam}
P(x_0=0, x_2=0\cond x_1=1, \theta)
\\
\frac{1}{nsam}
P(x_0=0, x_2=1\cond x_1=1, \theta)
\\
\frac{1}{nsam}
P(x_0=1, x_2=0\cond x_1=1, \theta)
\\
\frac{1}{nsam}
P(x_0=1, x_2=1\cond x_1=1, \theta)
\end{array}
}

\newcommand{\misscelltwo}[0]{
\begin{array}{c}
\frac{1}{nsam}
P(x_1=0\cond x_0=0,x_2=1, \theta)
\\
\frac{1}{nsam}
P(x_1=1\cond x_0=0,x_2=1,  \theta)
\end{array}
}


\newcommand{\td}[0]{{\TIL{d}}}
\newcommand{\rvtd}[0]{{\ul{\TIL{d}}}}
\newcommand{\tx}[0]{{\TIL{x}}}
\newcommand{\tmu}[0]{{\TIL{\mu}}}
\newcommand{\rvtx}[0]{{\ul{\TIL{x}}}}

\newcommand{\mlarr}[0]{\xrightarrow{\rm ML-fit}}
\newcommand{\lrarr}[0]{\xrightarrow{\rm LR-fit}}

\newcommand{\setprob}[3]
{{\begin{array}{c}S=\{#1\}
\\P(S)=#2\\ \haty(x^\s_S)=\$#3 K
\end{array}}}

\newcommand{\Gno}[0]{\xymatrix{\;\ar[r]|\parallel_G&}}
\newcommand{\Gyes}[0]{\xymatrix{\;\ar[r]_G&}}

\newcommand{\calypso}[0]{\ol{\caly}}

\newcommand{\SeqBdoorDef}[0]{
Suppose that we have access to data
that allows us to
estimate a probability
distribution
 $P(x^n, y, z^n)$.
Hence, the variables
$\rvx^n, \rvy, \rvz^n$ are
ALL observed (i.e, not hidden).
Then we say that the
the multinode
of ``covariates" $\rvz^n$
satisfies the
{\bf sequential backdoor (SBD) adjustment criterion}
relative to $(\rvx^n, \rvy)$
if for all $t\in\{0,1, \ldots, n-1\}$,

\begin{enumerate}
\item
$\rvy\perp\rvx_t|
\underbrace{(\rvx_0, \rvx_1, \ldots,\rvx_{t-1},
\rvz_0, \rvz_1, \ldots, \rvz_t)}
_{\text{Past of $\rvx_t$}}$
in $\call_{\rvx_{t}}
\cald_{\rvx_{t+1},\rvx_{t+2}
,\ldots, \rvx_{n-1}}G$.
\item
$\rvz_t \cap de(\rvx_t)=\emptyset$.
\end{enumerate}
}

\newcommand{\SeqBdoorClaim}[0]{
If $\rvz^n$ satisfies the
sequential backdoor criterion relative to
 $(\rvx^n, \rvy)$, then

\beq
P(y | \cald \rvx^n=x^n)=
\calq(y|x^n)
\;,
\eeq
where $\calq(y|x^n)$
is defined by
Eq.(\ref{def-q-y-xn-seqbdoor}).
}



\begin{document}
\title{CausalFitbit, an example of DEFT (DAG Extraction From Text)}
\date{ \today}
\author{Robert R. Tucci\\
        tucci@ar-tiste.com}
\maketitle
\vskip2cm
\section*{Abstract}
Previously,
we created a GitHub repo called  ``Mappa\_Mundi"  that contains a Python app for
doing causal DEFT (DAG Extraction from Text).
In this white paper, we
describe the Python app ``CausalFitbit"
that applies the  Mappa Mundi software  to 
a Fitbit dataset from Kaggle.
The CausalFitbit software 
is available as open source at  GitHub.

\newpage

Previously,
we created a GitHub repo called  ``Mappa\_Mundi" (Ref.\cite{github-mappa-mundi}) that contains a Python app for
doing causal DEFT (DAG Extraction from Text).

In this white paper, we
describe the Python app ``CausalFitbit"
that applies the  Mappa Mundi software  to 
a Fitbit dataset (Ref. \cite{fitbit-dataset}) from Kaggle.
The CausalFitbit software 
is available as open source at the GitHub repo
``CausalFitbit" (Ref.\cite{causal-fitbit}).

Note that the Mappa Mundi app uses LLMs 
to perform 2 distinct jobs: 
\begin{itemize}
\item doing {\bf sentence splitting} via a fine tuning of BERT called Openie6 (Ref.\cite{openie6})\footnote{Shameless
plug: You could also use my free software SentenceAx. SentenceAx (Ref.\cite{sentence-ax}) is a full rewrite of Openie6.}

\item calculating {\bf sentence similarities} via sBERT (Ref.\cite{sbert}).
\end{itemize}

CausalFitbit, on the other hand,
doesn't use LLMs at all (even though it uses the same algorithm as Mappa Mundi). How can this be? 

\begin{itemize}
\item
There is no sentence splitting to be done in CausalFitbit, because the sentences are already split; they are simple sentences from the outset.  
\item In CausalFitbit, we provide an analytical
formula for calculating sentence similarity. This will be explained better later on,
but basically, what we do is 
calculate the Z-distance between two ``sentences". A sentence in this case
is just something like $z=5$,
and the Z-distance between sentences
$z=5$ and $z=1$ is $|5-1|=4$.
\end{itemize} 

The simple sentences (ssents)
considered in the Mappa Mundi
repo are {\bf linguistic}, such as ``The ball is red". The ssents in 
CausalFitbit are {\bf symbolic},
such as $z=5$. In future,
we expect that we will be using 
the Mappa Mundi algorithm
with a combination of both
linguistic and symbolic ssents.

All necessary 
CausalFitbit 
operations are performed
by 3 jupyter notebooks
which we will describe in detail next.

\begin{enumerate}

\item {\bf {\color{red}{\tt heartrate\_data\_thinning.ipynb} } notebook}\footnote{Unless otherwise stated,
all jupyter  notebooks
are to be found in the folder entitled
{\tt jupyter\_notebooks}.}


The original dataset is available at  Kaggle (Ref. \cite{fitbit-dataset}) as a csv (comma separated values) file. The
dataset is fairly large ($\sim$ 85MB)
because it contains heartbeat (i.e., pulse) data sometimes every 10 seconds or so. 

In this notebook, we use Pandas to average the pulse data over each hour interval.  The resulting thinned dataset file
is just 160KB.  
 \item {\bf {\color{red}{\tt data\_preparation.ipynb}} notebook}


This notebook generates one 
 csv  file per patient (for a total of 33 patients). It
 stores those files in the folder named 
 {\tt patient\_csv\_records} 
 
 This notebook does a whole
 bunch of chores, most of which are  
 menial, but some aren't.
 One non-medial task it does is to add columns for ``velocity" features.
  
  
  Table \ref{tab-cfitbit-without-vel}
  shows a table illustrating a made-up, pedagogical 2 line dataset. Table \ref{tab-cfitbit-with-vel}
  shows the same
  dataset as Table \ref{tab-cfitbit-without-vel}, but after
  adding two more columns, f1Vel and f2Vel,
  for the velocities of the two features f1 and f2. 


\begin{table}[h!]
\centering
\begin{tabular}{|l|l|l|l|}
\hline
\rowcolor[HTML]{FFFFC7} 
id & datetime & f1 & f2 \\ \hline
1503960366 & 2016-04-12 00:00:00 & 3.1 & 1.8 \\ \hline
1503960366 & 2016-04-13 00:00:00 & 2.8 & 2.2 \\ \hline
\end{tabular}
\caption{Dataset without velocity columns.}
\label{tab-cfitbit-without-vel}
\end{table}


\begin{table}[h!]
\centering
\begin{tabular}{|l|l|l|l|l|l|}
\hline
\rowcolor[HTML]{FFFFC7} 
id & datetime & f1 & f1Vel & f2 & f2Vel \\ \hline
1503960366 & 2016-04-12 00:00:00 & 3.1 & nan & 1.8 & nan \\ \hline
1503960366 & 2016-04-13 00:00:00 & 2.8 & (2.8-3.1)/24 & 2.2 & (2.2-1.8)/24 \\ \hline
\end{tabular}
\caption{Dataset of Table \ref{tab-cfitbit-without-vel}
after adding 2 velocity columns f1Vel and f2Vel.}
\label{tab-cfitbit-with-vel}
\end{table}

As can be seen from Table
\ref{tab-cfitbit-with-vel},
to add a velocity column  for any feature of a Table, for instance f1,
we calculate $\Delta f1/\Delta t$, where 

$\Delta f1=$ the 
current value of f1
minus its previous value

$\Delta t=$ the current value of time in hours minus  the previous value of time.

This can all be done with one line of code using  the powerful
Pandas function {\tt diff()}.

\item {\bf {\color{red}
{\tt navigating\_patient\_records.ipynb}} notebook
}





\begin{itemize}

This notebook accomplishes
the following 3 tasks.
\item {\bf summarizing}

This step uses the class
{\tt PatientSimpRecord}. This class generates
a simp (simplified sentence) record file
for each csv patient 
record file in
the folder {\tt patient\_csv\_records}.
The class stores the resulting
files in the folder {\tt patient\_simp\_records}.

What we call summarizing 
is illustrated by Table \ref{tab-summarizing}.

For any feature (i.e., column) $f$,
let 

$\sigma_f=$ the standard 
deviation of the column $f$ (calculated
with the Pandas function {\tt std()})

$\av{f}=$ the average of the column $f$ (calculated with the Pandas function {\tt mean()})

Then

\beq
z(f) = \frac{f-\av{f}}{\sigma_f}
\label{eq-z-f}
\eeq

Wordifying means we replace 
a segment like {\tt f1=2.8}
by a segment like {\tt f1=2.8 \&z=.1}.
The latter looks like a ssent if you 
read it as  ``f1 equals 2.8 and $z$ equals 0.1".

An important feature of summarizing
is that, in a summarized file, the number of columns 
in different rows might be different,
because when summarizing, if there is missing information for 
a cell, we skip it.

\begin{table}[h!]
\centering
\begin{tabular}{|l|l|l|l|l|l|}
\hline
\rowcolor[HTML]{FFFFC7} 
id & datetime & f1 & f1Vel & f2 & f2Vel \\ \hline
1503960366 & 2016-04-13 00:00:00 & 2.8 & -.0125 & 2.2 & .0166 \\ \hline
\end{tabular}
\caption{This made-up single line
of a dataset would be replaced
by the following single line 
of a patient simp file: 
{\tt
f1= 2.8 \&z= .1<SEP>f1Vel= -.01 \&z= .3<SEP>f2= 2.2 \&z= .1<SEP>f2Vel= .016 \&z= .2}
}
\label{tab-summarizing}
\end{table}




\item {\bf DAG atlas}

This step is practically
identical
to its counterpart
step in the Mappa Mundi app.
In this step, we
use the class {\tt DagAtlas}
located in the file
{\tt mm\_DagAtlas}, to construct 
as Dag atlas, 
based on the 
patient files in folder
{\tt patient\_simp\_records}.
Note that we add the prefix ``mm\_" 
to all files coming verbatim from
Mappa Mundi app.

In building a DagAtlas,
we are confronted with
the decision problem of whether to accept or reject a bridge. The {\bf acceptance indicator functions} $A_{bridge}$
 for this
decision problem is as follows.
Let 

$simi()=$ {\bf similarity function}, 

$z(f)=$ defined by Eq.(\ref{eq-z-f})

$R=$ the {\bf z-radius},

$simi^*=$
{\bf similarity threshold}.

Then
\beq
simi(s_1, s_2)=
\left\{
\begin{array}{ll}
simi^* + 1 & \text{if }
|z(s_1)-z(s_2)| < R
\\
simi^* - 1 & \text{if }
|z(s_1)-z(s_2)| > R
\end{array}
\right.
\label{eq-simi-z-def}
\eeq

\beq
A_{bridge}=
\left\{
\begin{array}{ll}
1 & \text{if }
simi(s_1, s_2) > simi^*
\\
0 & \text{otherwise}
\end{array}
\right.
\label{eq-A-bridge}
\eeq


Eq.(\ref{eq-A-bridge})
is also used in Mappa Mundi, 
but Eq.(\ref{eq-simi-z-def}) is new. 


\item {\bf Visualizing}

This step is practically
identical
to its counterpart
step in the Mappa Mundi app.
In this step, we
use the class {\tt Dag}
(located in the file
{\tt mm\_Dag}) and graphviz,
to draw DAGs.



In visualizing our DAGs,
we are confronted with
a decision problem: whether to accept or reject an arrow. The {\bf acceptance indicator function}
$A_{arrow}$ for this decision
 problem is as follows.


\beq
A_{arrow}=
\left\{
\begin{array}{ll}
1 &
\text{if }
p_{acc}>p_{acc}^*, 
N_{acc}> N_{acc}^*
\\
0 & \text{otherwise}
\end{array}
\right. 
\label{eq-A-arrow}
\eeq
Eq.(\ref{eq-A-arrow})
is also used in Mappa Mundi.
There you will find
the definitions of $p_{acc}$ and $N$.
Only arrows that
satisfy $A_{arrow}=1$ are drawn.
The thresholds $p_{acc}^*$
an $N^*$ are
set by the
user.

\end{itemize}

\end{enumerate}






\bibliographystyle{plain}
\bibliography{references}
\end{document}
